\input harmstyle
\input musixtex
\nobarnumbers

\tit Melodické vedení hlasů, sekvence a závěry
\lead Základní pravidla, jak se pohybovat s hlasy v tónině

\section Sazby
Existují tři typy sazeb. V tomto případě platí, že obrázek vydá za tisíc slov.
\threecols{
	\subsection{Úzká}
	\twostaff{
		\Notes\zw{J}\relax\nextstaff
		\zw{ceg}\en
	}
	Vrchní hlasy jsou naskádány na sobě.
}{
	\subsection{Široká}
	\twostaff{
		\Notes\zw{J}\relax\nextstaff
		\zw{cgl}\en
	}
	Jednotlivé hlasy mají mezi sebou prostor. Jsou poskádané ob akordické tóny.
}{
	\subsection{Smíšená}
	\twostaff{
		\Notes\zw{J}\relax\nextstaff
		\zw{egl}\en
	}
	Hybrid mezi širokou a úzkou. Vyskytuje se nejméně často.
}

\section Typy pohybů
Jak se pohybují hlasy mezi jednotlivými funkcemi?
\fourcols{
	\subsection{Stranný}
	\onestaff{
		\Notes\zh{e}\ha{g}\zh{e}\ha{i}\en
	}
	Jeden hlas zůstává, druhý se hýbe.
}{
	\subsection{Rovný}
	\onestaff{
		\Notes\zh{f}\ha{h}\zh{g}\ha{k}\en
	}
	Oba hlasy se hýbou stejným směrem.
}{
	\subsection{Paralelní}
	\onestaff{
		\Notes\zh{e}\ha{g}\zh{f}\ha{h}\en
	}
	Hlasy se pohybují stejným směrem o stejně velký interval.
}{
	\subsection{Antiparalelní}
	\onestaff{
		\Notes\zh{f}\ha{h}\zh{e}\ha{j}\en
	}
	Hlasy se pohybují proti sobě.
}

\section Rozsah hlasů
Kde je možné psát jednotlivé hlasy v notové osnově?
\tableto{\hsize}{ll}{\crx
	Hlas & Rozsah\crlig
	Soprán (nejvyšší) & c\sub{1} -- a\sub{2}\crlig
	Alt & g -- c\sub{2}\crlig
	Tenor & c -- g\sub{1}\crlig
	Bas (nejnižší) & E -- c\sub{1} \cr
}

\bluebox{Jde o omezení vycházející z praxe, zpěvák s hlasem položeným v base vám nezazpívá f\sub{2}.}
\footer{Možná by bylo dobré se naučit rozsahy nazpaměť, avšak jsem na konci třeťáku a nikdy jsem to nepotřeboval.}

\section Obrat X poloha
Obrat je dán tónem v base. Poloha se určuje podle intervalu mezi basem a sopránem.
\twocols{%
	\bluebox{%
		\twostaff{
			\Notes\ha{J}\relax\nextstaff
			\zh{eg}\ha{j}\en
		}
		\footer{Tohle je kvintakord v oktávové poloze.}
	}
}{%
	\bluebox{%
		\twostaff{
			\Notes\ha{L}\relax\nextstaff
			\zh{ce}\ha{g}\en
		}
		\footer{Tohle je sextakord.}
	}
}

\section Zákazy
\begitems \style o
* Paralelní a antiparalelní kvinty a oktávy.
* Postup hlasu ve zvětšené sekundě.
\enditems

\section Klamný spoj
Spoj mezi D (5. stupněm) a VI (6. stupněm).
Prakticky jde o následek toho, že citlivý tón musí stoupat.
Všechny ostatní hlasy jdou protipohybem k basu.
\twocols{%
	\subsection{V dur}
	\twostaff{
		\NOtes\notefnk{D}\hl{N}\notefnk{VI}\hl{a}\relax\nextstaff
		\zh{dg}\ha{i}\zh{ce}\ha{j}\en
	}
}{%
	\subsection{V moll}
	\twostaff{
		\NOtes\notefnk{D}\hl{N}\notefnk{VI}\hl{_a}\relax\nextstaff
		\zh{dg}\ha{i}\zh{c_e}\ha{j}\en
	}
}

\section Předzdvojení
Pokud jsem v moll harmonické a mám spoj ze VI na D, musím použít tzv. předzdvojení.
Toto vychází z pravidla, že nesmíme jít s hlasem o zv. sekundu.
\twocols{%
	\redbox{
		\twostaff{
			\NOtes\notefnk{VI}\hl{_a}\notefnk{D}\hl{N}\relax\nextstaff
			\Red\zh{_hj}\ha{_l}\zh{=ik}\ha{n}\en
		}
		\footer{Postup ve zvětšené sekundě (tenor)}
	}
}{%
	\greenbox{
		\twostaff{
			\NOtes\notefnk{VI}\hl{_a}\notefnk{D}\hl{N}\relax\nextstaff
			\roffset{-1}{\zh{j}\fl{l}}\zh{j}\hl{l}\zh{=ik}\ha{n}\en
		}
		\footer{S předzdvojením}
	}
}

\section Sekvence
Sekvence je určitý motiv, který je stupňovitě přenášen o sekundu. Většinou je jeden až dva takty dlouhý.
\subsection{Jak vytvořit sekvenci?}
Tvoření je docela snadné.
Existuje jediné pravidlo, které musím dodržet (výjma pravidel, které už jsou dané, třeba jako zákaz paralelních kvint).
\redbox{Počáteční a koncový akord sekvence musí být v jiné poloze.}
Pojďme si to ukázat na příkladu:

\generalmeter\meterC
\generalsignature{2}
\setclef1\bass
\setstaffs1{2}
\startextract
	\NOtes\notefnk{T}\Red\ql{K}\Black\notefnk{T\sup{6}}\ql{M}\notefnk{S}\Red\ql{N}\Black|\Red\zq{fh}\qa{k}\Black\zq{dh}\qa{k}\Red\zq{dg}\qa{i}\Black\en
\endextract
\medskip
Jakmile mám vytvořený model, stačí ho už pak kopírovat/transponovat napříč akordy.

\generalmeter{\meterfrac34}
\startextract
	\NOtes\notefnk{T}\ql{K}\notefnk{T\sup{6}}\ql{M}\notefnk{S}\ql{N}|\zq{fh}\qa{k}\zq{dh}\qa{k}\zq{dg}\qa{i}\en\bar
	\NOtes\notefnk{II}\ql{L}\notefnk{II\sup{6}}\ql{N}\notefnk{D}\ql{a}|\zq{gi}\qa{l}\zq{ei}\qa{l}\zq{eh}\qa{j}\en\bar
	\NOtes\notefnk{III}\ql{M}\notefnk{III\sup{6}}\ql{a}\notefnk{VI}\ql{b}|\zq{hj}\qa{m}\zq{fj}\qa{m}\zq{fi}\qa{k}\en
\endextract

\section Závěry
Závěry se určujou podle posledních dvou funkcí v hudební větě. V osmitaktových cvičeních (které dává Bělík) jsou většinou závěry dva (na konci předvětí a závětí).

\generalmeter{\allabreve}%
\startpiece
\generalsignature{1}
	\NOtes\notefnk{T}\ha{G}\notefnk{T\sup{6}}\ha{I}|\zh{gi}\ha{k}\zh{dg}\ha{k}\en\bar
	\NOtes\notefnk{D\supsub{65}{43}}\zw{K}|\zh{dg}\ha{i}\zh{df}\ha{h}\en\bar
	\NOtes\notefnk{VI}\ha{L}\notefnk{II\supsub{6}{4}}\ha{J}|\zh{b}\roff{\zh{g}}\ha{g}\zh{ce}\ha{h}\en\bar
	\NOtes\notefnk{D\supsub{65}{43}}\zw{K}|\Red\zh{bd}\ha{g}\zh{ad}\ha{f}\Black\en\bar
	\NOtes\notefnk{T}\ha{N}\notefnk{T\supsub{6}{4}}\ha{K}|\zh{bd}\ha{g}\zh{bd}\ha{g}\en\bar
	\NOtes\notefnk{II\sup{6}}\ha{J}\notefnk{T\sup{6}}\ha{I}|\zh{ce}\ha{h}\zh{dg}\ha{k}\en\bar
	\NOtes\notefnk{II}\ha{H}\notefnk{D}\ha{K}|\zh{eh}\ha{j}\zh{df}\ha{h}\en\bar
	\NOtes\notefnk{T\supsub{65}{43}}\zw{G}|\Red\zh{eh}\ha{j}\zh{dg}\ha{i}\Black\en
\Endpiece

\bluebox{Za závěr se považujou dvě poslední funkce v části věty.}
A teď ta těžší část. Závěry určujeme podle několika kritérii.
\begitems \style o
* Končí-li závěr tónikou, je celý, jinak je poloviční.
\enditems

\subsection{Podle funkcí}
Závěry se dělí podle toho, jaké funkce obsahují.
\tableto{\hsize}{ll}{\crx
	Funkce & Závěr\crlig
	T-D nebo D-T & Autentický\crlig
	D-VI & Klamný\crlig
	T-S nebo S-T & Plagální\cr
}

\subsection{Melodicky dokonalý}
Soprán (vrchní hlas) musí končit základním tónem tóniky a musí k tónice postupovat stupňovitě (sekundou).

\generalmeter{}
\generalsignature{0}
\twocols{
	\greenbox{
		\twostaff{
			\NOtes\notefnk{D}\ha{N}\relax\nextstaff\zh{dg}\ha{i}\en\bar
			\NOtes\notefnk{T}\ha{J}\relax\nextstaff\zh{eg}\ha{j}\en
		}
		\footer{Závěr autentický celý; melodicky, harmonicky i rytmicky dokonalý.}
	}
}{
	\redbox{
		\twostaff{
			\NOtes\notefnk{D}\ha{N}\relax\nextstaff\zh{bd}\ha{g}\en\bar
			\NOtes\notefnk{T}\ha{J}\relax\nextstaff\zh{eg}\ha{j}\en
		}
		\footer{Závěr autentický celý; {\bf melodicky zeslabený}, harmonicky i rytmicky dokonalý.}
	}
}

\subsection{Harmonicky dokonalý}
Poslední akord je v oktávové poloze a poslední dva akordy nejsou v obratu (mají základní tón v base).
\twocols{
	\greenbox{
		\twostaff{
			\NOtes\notefnk{D}\ha{N}\relax\nextstaff\zh{dg}\ha{i}\en\bar
			\NOtes\notefnk{T}\ha{J}\relax\nextstaff\zh{eg}\ha{j}\en
		}
		\footer{Závěr autentický celý; melodicky, harmonicky i rytmicky dokonalý.}
	}
}{
	\redbox{
		\twostaff{
			\NOtes\notefnk{D\sup{87}}\zw{N}\relax\nextstaff\zh{bd}\ha{g}\zh{bd}\ha{f}\en\bar
			\NOtes\notefnk{T}\ha{J}\relax\nextstaff\zh{eg}\ha{j}\en
		}
		\footer{Závěr autentický celý; melodicky zeslabený, {\bf harmonicky zeslabený}, rytmicky dokonalý }
	}
}

\subsection{Rytmicky dokonalý}
Poslední akord je na těžké době.
\twocols{
	\greenbox{
		\twostaff{
			\NOtes\notefnk{D}\ha{N}\relax\nextstaff\zh{dg}\ha{i}\en\bar
			\NOtes\notefnk{T}\ha{J}\relax\nextstaff\zh{eg}\ha{j}\en
		}
		\footer{Závěr autentický celý; melodicky, harmonicky i rytmicky dokonalý.}
	}
}{
	\redbox{
		\generalmeter{\meterfrac34}
		\twostaff{
			\NOtes\notefnk{D}\ql{N}\notefnk{VI}\ha{a}\relax\nextstaff\zq{bd}\qa{g}\zh{cg}\ha{j}\en
		}
		\footer{Závěr klamný; melodicky dokonalý, harmonicky nedokonalý, {\bf rytmicky zeslabený}}
	}
}

Pokud je nějaká z podmínek nesplněna, je závěr v daném \uv{oboru} zeslabený.

\bye
